%! Author = EricW
%! Date = 2024/11/22

% Preamble
\documentclass[11pt]{ctexart}

% Packages
\usepackage{amsmath}
\usepackage{amssymb}
\usepackage{tabularx}
\usepackage{graphics}

% Document
\begin{document}

    \section{T2}
    
    \subsection{物理模型的假设}
在水下成像模型 $I(x) = J(x) \cdot t(x) + B$ 中,
核心的两个变量$t(x)$(光传输函数)和$t(x)$(环境光分量)是影响水下图像质量的关键因素。
为了对退化类型进行分析,需要对$t(x)$和$B$进行合理的假设,使得模型具有物理可解释性,并能应用于不同退化场景。


    \textbf{光传输函数}$\boldsymbol{t(x)}$


    在水下图像退化模型的构建中,我们对光传输函数$t(x)$和环境光$B$作出了简化假设,旨在通过物理规律来描述水下成像过程。
    首先,我们假设光衰减遵循指数衰减规律,即\[t(x)=e^{-\beta d(x)}\]
    其中,
    \begin{description}
        \item[$\beta$]:光衰减系数,描述水体对光的吸收和散射能力;
        \item[$d(x)$]:光传播路径长度(光从物体表面到摄像机的传播距离)。
    \end{description}
    这一假设源自于Beer-Lambert定律,该定律广泛应用于光的传播过程中,尤其是在吸收和散射介质中。
    不同波长的光在水中的吸收和散射系数$\beta $不同,红光(长波长)衰减最快,绿光次之,蓝光衰减最慢,
    这也解释了水下图像中颜色偏移现象。
    而光的衰减程度与传播距离$d(x)$成正比。在浅水区,光的衰减较小;而在深水区,光经过更长的路径,因此$t(x)$会显著降低。


    \textbf{背景光}$\boldsymbol{B}$


    水下背景光主要来源于水体悬浮颗粒对自然光的散射和水体对光的吸收,呈现为背景光强度随着距离逐渐增加并最终趋于饱和的样式。
    于是我们假设\[B=B_0(1-e^{-\gamma d(x)})\]
    其中,
    \begin{description}
        \item [$B_0$]:背景光的最大强度;
        \item [$\gamma$]:描述背景光衰减速率的水体浑浊度参数;
        \item [$d(x)$]:光传播路径长度(水深)
    \end{description}
    与此同时,受到水体浑浊度$\gamma $的影响,水体浑浊度越高,背景光强度增加得越快;
    而水体越深,$d(x)$越大,此时$B$越趋近于$B_0$,背景光成为主要光源,与现实情况一致,水体浅亦然。
    此外,我们还假设水下图像的每个颜色通道(红,绿,蓝)受到不同波长的$t_{r} (x)$,$t_{g} (x)$,$t_{b} (x)$和$B_r$,$B_g$,$B_b$的影响。

    \subsection{针对退化类型的分析}
    \subsubsection{Color Cast}
    颜色偏移是由于不同波长的光的吸收差异所引起的,我们使用了吸收系数$\beta $来区分不同波长的光的吸收差异。
    \begin{description}
        \item [红光]:长波长,吸收最严重,传播距离最短。
        \item [绿光]:中波长,中等吸收,传播距离较长。
        \item [蓝光]:短波长,吸收最小,传播距离最长。
    \end{description}
    红光的快速衰减将导致水下图像呈绿色或蓝色,即产生了颜色偏移现象。


    结合先前做出的光衰减函数假设,我们将颜色偏移的成像模型定义为\[I_c(x) = J_c(x) \cdot e^{-\beta_c d(x)} + B_c,\ c\in \{R,G,B\}\],
    其中$\beta_c$为不同颜色的吸收系数,通常来说$\beta_R \gg \beta_G > \beta_B $,且$\beta_G$与$\beta_B$约等于0。
    在浅水区域,$d(x)$较小,$\beta_c$的影响较弱,$\beta_Rd(x)$,$\beta_Gd(x)$,$\beta_Bd(x)$差异小,颜色偏移不显著;\\
    在深水区域,$d(x)$较大,$\beta_c$的影响较强,$e^{-\beta_Rd(x)}$趋于零,红光急速衰减,颜色偏移显著。\\
    此外,$B_c$的值由背景光的波长依赖特性决定。通常,蓝光和绿光在水中传播更远,因此$B_{0_B} > B_{0_G} > B_{0_R}$,
    在同一深度下,这也导致了背景光中蓝绿光的比例较高,进一步加剧了图像的蓝绿色调。

    \subsubsection{Low Light}
    低光照主要受水深($d(x)$)的影响。\\
    由先前假设我们可以得到光衰减函数$t(x)=e^{-\beta d(x)}$,当$d(x)$很大时,$t(x)$趋于0,导致物体反射的光几乎完全消失,
    而使得了背景光$B$占据了主导作用。由假设\[B=B_0(1-e^{-\gamma d(x)})\]在浅水区($d(x)$小)时,$B$较小,但随着深度($d(x)$增大)的增加,
    $B$逐渐变大,最终趋于饱和。此外,受到水体浑浊的影响,水体浑浊度($\gamma$)越高,背景光强度增加的越快,在较浅的深度中也可能使得$B$较大而占据了主导作用。\\
    显然,在低亮度条件下我们得到的平均亮度$L_{mean}$将与不同种类光(环境光,散射光等)的平均亮度$I(x)$接近。由亮度对比度公式
    \[L_{std} = \sqrt{\frac{1}{N} \sum_{x} (I(x) - L_{mean})^2}\]
    可知$L_{std}$较小,这表示着图像的光照分布越均匀,缺乏对比度,导致细节丧失。

    \subsubsection{Blur}
    模糊主要受到散射效应的影响。而散射效应通过影响模糊核$k(x)$而来影响图像的清晰度。\\
    由先前的假设我们可以得到模糊的退化方程为\[I(x) = (J(x) \cdot e^{-\beta d(x)})*k(x) + B\]
    且我们假设模糊核为\[k(x) = \frac{1}{2\pi\sigma^2} e^{-\frac{x^2 + y^2}{2\sigma^2}}\]
    模糊核的$\sigma$越大,模糊越显著。\\
    此外,当深度大的时,$t(x)$小,此时图像主要受$k(x)$的影响,这也说明了模糊可以与低光照现象叠加而影响图像的清晰度。
    \subsection{综合分析与比较}
    综上我们可以知道图片的清晰度主要受颜色偏移、低光照和模糊的影响。而颜色偏移主要是受到光衰减函数$t(x)$的影响,低光照主要是受到水深$d(x)$的影响,
    而模糊主要是受到散射效应(模糊核)的影响。具体由表格表示如下。

\begin{table}[ht]
\centering
\begin{tabularx}{\textwidth}{|X|X|X|}
\hline
\textbf{退化类型} & \textbf{定量分析核心参数} & \textbf{退化原因总结} \\ \hline
颜色偏移     & $\beta_c, B_c$ & 波长依赖的光衰减导致红光快速衰减,背景光中的蓝绿光占主导,引起显著的颜色偏移。 \\ \hline
低光照       & $\beta, B, L_{std}$      & 指数衰减函数 $e^{-\beta d(x)}$ 导致整体光强快速下降,背景光逐渐占主导,形成低光照问题。 \\ \hline
模糊         & $\sigma$               & 散射效应通过高斯模糊核 $k(x)$ 降低物体的高频细节,背景光影响对比度,导致模糊。 \\ \hline
\end{tabularx}
\caption{退化类型的综合分析}
\end{table}






\end{document}