%! Author = EricW
%! Date = 2024/11/22

% Preamble
\documentclass[11pt]{ctexart}

% Packages
\usepackage{amsmath}
\usepackage{amssymb}
\usepackage{tabularx}
\usepackage{graphicx}

% Document
\begin{document}

    \section{T2}

    \subsection{Assumptions of the Physical Model}
    In the underwater imaging model \( I(x) = J(x) \cdot t(x) + B \),
    the two core variables \( t(x) \) (light transmission function) and \( B \) (ambient light component) are pivotal to the quality of underwater images.
    To analyze types of degradation, it's necessary to make reasonable assumptions about \( t(x) \) and \( B \), ensuring the model is physically interpretable and applicable across various degradation scenarios.

    \subsubsection{Light Transmission Function \( \boldsymbol{t(x)} \)}
    When constructing the underwater image degradation model, we've made simplified assumptions about the light transmission function \( t(x) \) and ambient light \( B \), aiming to describe the underwater imaging process through physical laws.
    Firstly, we assume that light attenuation follows an exponential decay law, that is:
    \[ t(x) = e^{-\beta d(x)} \]
    where,
    \begin{description}
        \item [$\beta$] is the light attenuation coefficient, describing the water's capacity to absorb and scatter light;
        \item [$d(x)$] is the light propagation path length (the distance from the object's surface to the camera).
    \end{description}
    This assumption is grounded in the Beer-Lambert law, widely applied in the propagation of light, particularly in media that absorb and scatter light.
    Different wavelengths of light have varying absorption and scattering coefficients \( \beta \) in water, with red light (longer wavelengths) attenuating the fastest, followed by green light, and blue light attenuating the slowest,
    which also explains the color shift phenomenon in underwater images.
    The degree of light attenuation is directly proportional to the propagation distance \( d(x) \). In shallow waters, light attenuation is minimal; however, in deeper waters, where light travels a longer path, \( t(x) \) significantly decreases.

    \subsubsection{Ambient Light \( \boldsymbol{B} \)}
    Underwater ambient light primarily stems from the scattering of natural light by suspended particles in the water and the water's absorption of light, characterized by an increasing background light intensity with distance, eventually reaching saturation.
    Thus, we assume:
    \[ B = B_0(1 - e^{-\gamma d(x)}) \]
    where,
    \begin{description}
        \item [$B_0$] is the maximum intensity of the ambient light;
        \item [$\gamma$] is the turbidity parameter of the water body, describing the rate at which the ambient light decays;
        \item [$d(x)$] is the light propagation path length (water depth).
    \end{description}
    At the same time, the turbidity \( \gamma \) of the water affects the rate at which the ambient light intensity increases; the higher the turbidity, the faster the increase in ambient light intensity.
    Furthermore, the deeper the water, the greater \( d(x) \) becomes, causing \( B \) to approach \( B_0 \), making the ambient light the primary light source, consistent with real-world observations.

    \subsection{Analysis of Degradation Types}
    \subsubsection{Color Cast}
    Color cast is caused by the differential absorption of light at different wavelengths, which we distinguish using the absorption coefficient \( \beta \).
    \begin{description}
        \item [Red Light] Long wavelengths, most severely absorbed, shortest propagation distance.
        \item [Green Light] Medium wavelengths, moderately absorbed, longer propagation distance.
        \item [Blue Light] Short wavelengths, least absorbed, longest propagation distance.
    \end{description}
    The rapid attenuation of red light leads to an underwater image that appears green or blue, resulting in a color cast.

    Incorporating the previously assumed light attenuation function, we define the imaging model for color cast as:
    \[ I_c(x) = J_c(x) \cdot e^{-\beta_c d(x)} + B_c, \quad c \in \{R, G, B\} \]
    where \( \beta_c \) is the absorption coefficient for different colors. Typically, \( \beta_R \gg \beta_G > \beta_B \), with \( \beta_G \) and \( \beta_B \) being close to zero.
    In shallow water areas, where \( d(x) \) is small, the effect of \( \beta_c \) is weak, and the differences between \( \beta_R d(x) \), \( \beta_G d(x) \), and \( \beta_B d(x) \) are minimal, making the color cast inconspicuous.
    In deeper water areas, where \( d(x) \) is large, the effect of \( \beta_c \) is more pronounced.

    \subsubsection{Low Light}
    Low light is primarily influenced by water depth (\( d(x) \)).
    From our previous assumptions, we derive the light attenuation function \( t(x) = e^{-\beta d(x)} \). When \( d(x) \) is large, \( t(x) \) approaches zero, causing the light reflected by objects to almost completely vanish,
    allowing the ambient light \( B \) to dominate. According to our assumption:
    \[ B = B_0(1 - e^{-\gamma d(x)}) \]
    In shallow waters (\( d(x) \) is small), \( B \) is relatively low, but as depth (\( d(x) \)) increases,
    \( B \) gradually increases, eventually reaching saturation. Additionally, higher water turbidity (\( \gamma \)) leads to a faster increase in ambient light intensity, potentially causing \( B \) to dominate at shallower depths as well.
    Clearly, under low light conditions, the average brightness \( L_{mean} \) we obtain will be close to the average brightness \( I(x) \) of different types of light (ambient light, scattered light, etc.). From the brightness contrast formula:
    \[ L_{std} = \sqrt{\frac{1}{N} \sum_{x} (I(x) - L_{mean})^2} \]
    a lower \( L_{std} \) indicates a more uniform light distribution across the image, lacking contrast and leading to a loss of detail.

    \subsubsection{Blur}
    Blur is mainly affected by scattering effects, which impact image clarity through the blurring kernel \( k(x) \).
    From our previous assumptions, the degradation equation for blur is:
    \[ I(x) = (J(x) \cdot e^{-\beta d(x)}) * k(x) + B \]
    and we assume the blurring kernel to be:
    \[ k(x) = \frac{1}{2\pi\sigma^2} e^{-\frac{x^2 + y^2}{2\sigma^2}} \]
    A larger \( \sigma \) in the blurring kernel indicates more significant blur.
    Moreover, when depth is great, \( t(x) \) is small, and the image is primarily influenced by \( k(x) \), which also explains how blur can compound with low light conditions to affect image clarity ("compound" could be used to indicate the cumulative effect on image clarity).

    \subsection{Comprehensive Analysis and Comparison}
    Summarizing the above, image clarity is primarily affected by color cast, low light, and blur.
    Color cast is mainly influenced by the light attenuation function \( t(x) \), low light is primarily affected by water depth \( d(x) \),
    and blur is mainly influenced by scattering effects (the blurring kernel). The following table provides a detailed analysis:

    \begin{table}[ht]
    \centering
    \begin{tabularx}{\textwidth}{|>{\centering\arraybackslash}X|>{\centering\arraybackslash}X|>{\centering\arraybackslash}X|}
    \hline
    \textbf{Type of Degradation} & \textbf{Key Parameters for Quantitative Analysis} & \textbf{Summary of Degradation Causes} \\ \hline
    Color Cast & $\beta_c, B_c$ & Wavelength-dependent light attenuation leads to rapid decay of red light, with blue and green light from the background dominating, causing significant color shifts. \\ \hline
    Low Light & $\beta, B, L_{std}$ & The exponential decay function $e^{-\beta d(x)}$ causes a rapid decrease in overall light intensity, allowing background light to gradually dominate, resulting in low light conditions. \\ \hline
    Blur & $\sigma$ & Scattering effects reduce high-frequency details of objects through the Gaussian blurring kernel $k(x)$, and background light affects contrast, leading to blur. \\ \hline
    \end{tabularx}
    \caption{Comprehensive Analysis of Degradation Types}
    \end{table}

\end{document}


